\documentclass[12pt, a4paper]{article}
\usepackage[utf8]{inputenc}
\usepackage[brazil]{babel}
\usepackage{graphicx}
\usepackage{geometry}
\usepackage{listings}
\usepackage{color}
\usepackage{hyperref}

\geometry{
 a4paper,
 total={170mm,257mm},
 left=20mm,
 top=20mm,
}

% Configuração para códigos
\definecolor{codegreen}{rgb}{0,0.6,0}
\definecolor{codegray}{rgb}{0.5,0.5,0.5}
\definecolor{codepurple}{rgb}{0.58,0,0.82}
\definecolor{backcolour}{rgb}{0.95,0.95,0.92}

\lstdefinestyle{mystyle}{
    backgroundcolor=\color{backcolour},   
    commentstyle=\color{codegreen},
    keywordstyle=\color{magenta},
    numberstyle=\tiny\color{codegray},
    stringstyle=\color{codepurple},
    basicstyle=\ttfamily\footnotesize,
    breakatwhitespace=false,         
    breaklines=true,                 
    captionpos=b,                    
    keepspaces=true,                 
    numbers=left,                    
    numbersep=5pt,                  
    showspaces=false,                
    showstringspaces=false,
    showtabs=false,                  
    tabsize=2
}
\lstset{style=mystyle}

\title{\textbf{Relatório Técnico: E-Community}}
\author{
    \textbf{Alunos:} \\
    Edgard de Paiva \\
    Lucas Dias \\
    Robson Duarte \\
    \\
    \textbf{Professor:} Ilo Rivero \\
    \textbf{Matéria:} Desenvolvimento Móvel
}
\date{\today}

\begin{document}

\maketitle
\newpage

\tableofcontents
\newpage

\section{Introdução}
O projeto \textbf{E-Community} é uma aplicação móvel desenvolvida com o objetivo de promover a sustentabilidade e a reciclagem dentro das comunidades. A plataforma serve como um hub central onde usuários podem aprender sobre práticas ecológicas, descartar resíduos corretamente e comercializar produtos sustentáveis.

O aplicativo visa resolver o problema da falta de informação e incentivo para a reciclagem, conectando tecnologia de ponta (Inteligência Artificial) com necessidades ambientais cotidianas.

\section{Tecnologias Utilizadas}
Para o desenvolvimento deste projeto, utilizamos uma stack moderna e robusta focada em desenvolvimento móvel multiplataforma:

\begin{itemize}
    \item \textbf{Linguagem:} Dart
    \item \textbf{Framework Mobile:} Flutter (para Android e iOS)
    \item \textbf{Backend as a Service (BaaS):} Google Firebase
    \begin{itemize}
        \item \textbf{Firebase Authentication:} Gestão de usuários e login seguro.
        \item \textbf{Firestore Database:} Banco de dados NoSQL para armazenamento de posts, produtos e perfis.
        \item \textbf{Cloud Functions:} Backend serverless para lógica de negócios complexa e integração com IA.
    \end{itemize}
    \item \textbf{Inteligência Artificial:} Google Gemini API (via Vertex AI / Google AI Studio)
    \begin{itemize}
        \item Utilizada para o "EcoMestre", um assistente virtual que tira dúvidas sobre reciclagem.
    \end{itemize}
\end{itemize}

\section{Arquitetura e Estrutura do Projeto}
O projeto segue uma arquitetura baseada em componentes e funcionalidades, organizada para facilitar a manutenção e escalabilidade.

\subsection{Estrutura de Pastas}
A estrutura principal do código fonte (\texttt{lib/}) é organizada da seguinte forma:

\begin{itemize}
    \item \texttt{main.dart}: Ponto de entrada da aplicação e configuração de temas e rotas.
    \item \texttt{models/}: Classes de dados que representam as entidades do sistema.
    \begin{itemize}
        \item \texttt{product\_model.dart}: Definição de produtos do marketplace.
        \item \texttt{post\_model.dart}: Definição de postagens da comunidade.
        \item \texttt{user\_model.dart}: Dados do perfil do usuário.
    \end{itemize}
    \item \texttt{screens/}: Telas da aplicação (UI).
    \begin{itemize}
        \item \texttt{auth/}: Telas de Login e Registro.
        \item \texttt{marketplace/}: Listagem e edição de produtos sustentáveis.
        \item \texttt{ai\_assistant/}: Interface de chat com o assistente EcoMestre.
        \item \texttt{feed/}: Feed de notícias e dicas ecológicas.
    \end{itemize}
    \item \texttt{services/}: Camada de serviços para comunicação com APIs e Firebase.
\end{itemize}

\section{Principais Funcionalidades}

\subsection{1. Autenticação e Perfil}
O sistema permite que usuários criem contas seguras. Utilizamos o Firebase Auth para gerenciar sessões. Cada usuário possui um perfil onde pode ver seus posts e produtos cadastrados.

\subsection{2. Marketplace Sustentável}
Uma área dedicada onde usuários podem anunciar itens recicláveis, artesanatos ecológicos ou produtos usados.
\begin{itemize}
    \item Funcionalidade de CRUD (Criar, Ler, Atualizar, Deletar) completa para produtos.
    \item Upload de imagens de produtos.
\end{itemize}

\subsection{3. EcoMestre (Assistente de IA)}
Um dos grandes diferenciais do projeto é o assistente virtual integrado.
\begin{itemize}
    \item Utiliza a API do \textbf{Google Gemini 1.5} para processar linguagem natural.
    \item O assistente foi treinado (via System Prompt) para atuar como um "Consultor Ecológico", respondendo apenas perguntas relacionadas à sustentabilidade.
    \item A comunicação é feita através de uma \textbf{Cloud Function} segura, garantindo que a chave de API não fique exposta no aplicativo cliente.
\end{itemize}

\subsection{4. Feed da Comunidade}
Um espaço social onde dicas, notícias sobre coletas e eventos de reciclagem podem ser compartilhados entre os usuários.

\section{Implementação da IA (Detalhes Técnicos)}
A integração com a IA foi realizada utilizando \texttt{Firebase Cloud Functions} (2ª Geração).

\begin{lstlisting}[language=JavaScript, caption=Exemplo da Cloud Function de IA]
// Trecho simplificado da função backend
exports.getGeminiResponse = onCall({ cors: true }, async (request) => {
    const userPrompt = request.data.text;
    
    const systemInstruction = "Voce e um especialista em ecologia...";
    const model = genAI.getGenerativeModel({ model: "gemini-1.5-flash" });
    
    const result = await model.generateContent(systemInstruction + userPrompt);
    return { text: result.response.text() };
});
\end{lstlisting}

Essa arquitetura garante segurança, pois a API Key do Gemini fica armazenada apenas no servidor (variáveis de ambiente), e permite atualizar a lógica da IA sem precisar lançar uma nova versão do aplicativo na loja.

\section{Conclusão}
O E-Community demonstra como tecnologias móveis modernas podem ser aplicadas para resolver problemas reais. A integração bem-sucedida do Flutter com o ecossistema Google (Firebase e Gemini) resultou em uma aplicação fluida, segura e inteligente, capaz de engajar a comunidade na causa da sustentabilidade.

\end{document}
